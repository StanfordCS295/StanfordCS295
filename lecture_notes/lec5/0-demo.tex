\usepackage{shared/cs45}

\title{CS 45, Lecture X}
\subtitle{Lecture Title}
\date{Winter 2023}
\author{Akshay Srivatsan, Ayelet Drazen, Jonathan Kula}

\begin{document}

\maketitle

\frame{\titlepage}

\begin{frame}
  \frametitle{Outline}
  \tableofcontents[hidesubsections]
\end{frame}

\section{Text Editing: An Overview}
\subsection{Rich Text}

When we think about editing a document, we usually think of doing that in 
a \em{rich text editor}, something like Word or Google Docs.

Rich text is for \em{humans communicating with humans}-- its elements
are structured around elements of prose, such as words, paragraphs,
headings, etc., and its features are centered around making consuming
written text easier for humans-- things like varying fonts, emphasizing
text with bold, italic, or underline, the ability to insert pictures
or other multimedia, and so on and so forth.

However, while this information is helpful and sometimes really necessary
in human-to-human communication, it's unnecessary and gets in the way when
we're desiring to communicate with a computer (or give it instructions).
This is why we use \em{plain text} for computers!

\subsection{Plain Text Editors}

Lots of different kinds of programs have been developed to edit plain text--
in fact, it's really one of the core affordances a computer offers. Some
plain text editors, like Windows' Notepad or macOS' TextEdit are extremely
basic, and fulfill the mantle of a plain text editor with no frills.
However, many of these 


\end{document}
